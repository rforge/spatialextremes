\documentclass{article}
\usepackage{amsmath}
\usepackage{amsfonts}
\begin{document}
Max-stable processes are the extension of the multivariate extreme
value theory to the infinite dimensional setting. More precisely
consider a stochastic process $\{Y(x)\}$, $x \in \mathbb{R}^d$, having
continuous sample paths. Then the limiting process
\begin{equation}
  \label{eq:limitProcess}
  \left\{ \max_{i=1, \ldots, n}
    \frac{Y_i(x) -b_n(x)}{a_n(x)} \right\}_{x \in \mathbb{R}^d}
  \longrightarrow \{Z(x)\}_{x \in \mathbb{R}^d}, \qquad n \to \infty
\end{equation}
where $Y_i$ are independent replications of $Y$, $a_n(x)> 0$ and
$b_n(x) \in \mathbb{R}$ are sequences of continuous functions and the
limiting process $Z$ is assumed to be non degenerate. de Haan [1994]
shows that the class of the limiting processes $Z(x)$ corresponds to
that of max-stable processes and hence emphasizes on their use to
model spatial extremes.

Interestingly there are two different ways of charactaresing
max-stable processes: these are known as the spectral
characterisations. An especially useful special case of the
characterisation of de Haan [1984], is
\begin{equation}
  \label{eq:deHaanCarac}
  Z(x) = \max_{i\geq 1} \xi_i f(x - U_i)
\end{equation}
where $\{(\xi_i, U_i)\}_{i \geq 1}$ are the points of a Poisson point
process on $(0, \infty] \times \mathbb{R}^d$ with intensity measure
$\mbox{d$\Lambda$}(\xi,u) = \xi^{-2} \mbox{d$\xi$} \mbox{du}$, $f$ is
a probability density function on $\mathbb{R}^d$. The process $Z$
defined above is a max-stable process with unit Frechet
margins. Taking $f$ as the multivariate Normal density with zero mean
and covariance matrix $\Sigma$ gives the Smith model [Smith, 1990] for
which the bivariate distribution function is
\begin{equation}
  \label{eq:smithCDF}
  \Pr[Z(x_1) \leq z_1, Z(x_2)
  \leq z_2] = \exp\left\{- \frac{1}{z_1} \Phi\left(\frac{a}{2} +
      \frac{1}{a} \log \frac{z_1}{z_2} \right) - \frac{1}{z_2}
    \Phi\left(\frac{a}{2} + \frac{1}{a} \log \frac{z_2}{z_1} \right)
  \right\},
\end{equation}
where $\Phi$ is the standard normal distribution function and $a^2 =
(x_1 - x_2)^T \Sigma^{-1} (x_1 - x_2)$, $x_1, x_2 \in \mathbb{R}^d$.

Another very useful spectral characterisation for unit Frechet
max-stable processes is [Schlather, 2002]
\begin{equation}
  \label{eq:schlatherCarac}
  Z(x) = \max_{i \geq 1} \xi_i Y_i(x),
\end{equation}
where $\{\xi_i\}_{i \geq 1}$ are the points of a Poisson point process
on $(0, \infty]$ with intensity measure $\mbox{d$\Lambda$}(\xi) =
\xi^{-2} \mbox{d$\xi$}$ and $Y_i$ are independent replications of a
positive stochastic process having continuous sample paths $Y$ such
that $\mathbb{E}[Y(x)] = 1$ for all $x \in \mathbb{R}^d"$.

Currently there are several useful models based on Schlather's
characterisation. The first model, the Schlather's model, is to use
$Y(x) = \sqrt{2 \pi} \max \{0, \varepsilon(x)\}$, where $\varepsilon$
is a standard Gaussian process. This leads to the bivariate
distribution function
\begin{equation}
  \label{eq:schlatherCDF}
  \Pr[Z(x_1) \leq z_1, Z(x_2) \leq z_2] = \exp\left[-\frac{1}{2}
    \left(\frac{1}{z_1} + \frac{1}{z_2} \right) \left(1 + \sqrt{1 -
        \frac{2 \{1 + \rho(h) \} z_1 z_2}{(z_1 +z_2)^2}} \right)
  \right],
\end{equation}
Another possibility is to take $Y(x) = \exp\{ \sigma \varepsilon(x) -
\sigma^2 / 2\}$, where $\sigma > 0$. This is known as the Geometric
gaussian model for which the bivariate distribution is similar to the
Smith model with $a^2 = 2 \sigma^2 \{1 - \rho(h)\}$. A last
possibility, which is a generalisation of the geometric Gaussian
model, is to take $Y(x) = \exp\{\varepsilon(x) - \gamma(x)\}$, where
$\varepsilon$ is a zero mean Gaussian process having stationary
increments and (semi)variogram $\gamma$ such that $\varepsilon(o) = 0$
almost surely. Its bivariate distribution function is again similar to
the Smith model with $a^2 = 2 \gamma(x_1 - x_2)$.
\end{document}
