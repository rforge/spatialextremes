\documentclass{article}
\usepackage{a4wide}
\usepackage{amsmath}

\begin{document}

\section{The Smith Characterisation}
\label{sec:smith-char}

The Smith characterisation of a max-stable process is given by:

\begin{equation}
  \label{eq:smith}
  \Pr[Z_1 \leq z_1, Z_2 \leq z_2] = \exp\left[-\frac{1}{z_1} \Phi
    \left(\frac{a}{2} + \frac{1}{a} \log \frac{z_2}{z_1} \right) -
    \frac{1}{z_2} \Phi \left(\frac{a}{2} + \frac{1}{a}
      \log\frac{z_1}{z_2} \right) \right]
\end{equation}
where $\Phi$ is the standard normal cumulative distribution function
and, for two location \#1 and \#2  
\begin{equation*}
  a^2 = \Delta x^T \Sigma^{-1} \Delta x \quad \text{and} \quad 
  \Sigma = 
  \begin{bmatrix}
    cov_{11} & cov_{12}\\
    cov_{12} & cov_{22}
  \end{bmatrix}
\end{equation*}
where $\Delta x$ is the distance vector between location \#1 and
location \#2.

\subsection{Useful quantities}
\label{sec:usefull-quantities}

Computation of the density as well as the gradient of the density is
not difficult but ``heavy'' though. For computation facilities and to
help readers, we define:
\begin{eqnarray}
  \label{eq:1}
  c_1 = \frac{a}{2} + \frac{1}{a} \log \frac{z_2}{z_1} \quad
  \text{and} \quad
  c_2 = \frac{a}{2} + \frac{1}{a} \log \frac{z_1}{z_2}
\end{eqnarray}
From these definitions, we note that $c_1 + c_2 = a$.

\subsection{Density computation}
\label{sec:density-computation}

From \eqref{eq:smith}, we note the standard normal distribution
appears. Consequently, we need to compute its derivatives in $c_1$ and
$c_2$.
\begin{eqnarray}
  \label{eq:2}
  \frac{\partial c_1}{\partial z_1} = \frac{1}{a} \left(-
    \frac{z_2}{z_1^2} \frac{z_1}{z_2} \right) = -\frac{1}{az_1} &\qquad&
  \frac{\partial c_1}{\partial z_2} = \frac{1}{a} \frac{1}{z_1}
  \frac{z_1}{z_2} = \frac{1}{az_2}\\
  \frac{\partial c_2}{\partial z_1} = - \frac{\partial c_1}{\partial z_1}
  = \frac{1}{az_1} &\qquad&
  \frac{\partial c_2}{\partial z_2} = - \frac{\partial c_1}{\partial z_2}
  = - \frac{1}{az_2}  
\end{eqnarray}

As the normal distribution appears in the Smith characterisation, the
following quantities will be useful:
\begin{eqnarray}
  \label{eq:6}
  \frac{\partial \Phi(c_1)}{\partial z_1} = \frac{\partial
    \Phi(c_1)}{\partial c_1}
  \frac{\partial c_1}{\partial z_1} =  -\frac{\varphi(c_1)}{az_1} &\qquad&
  \frac{\partial \Phi(c_1)}{\partial z_2} = \frac{\partial
    \Phi(c_1)}{\partial c_1}
  \frac{\partial c_1}{\partial z_2} =  \frac{\varphi(c_1)}{az_2}\\
  \frac{\partial \Phi(c_2)}{\partial z_1} = \frac{\partial \Phi(c_2)}{\partial c_2}
  \frac{\partial c_2}{\partial z_1} =  \frac{\varphi(c_2)}{az_1} &\qquad&
  \frac{\partial \Phi(c_2)}{\partial z_2} = \frac{\partial \Phi(c_2)}{\partial c_2}
  \frac{\partial c_2}{\partial z_2} =  -\frac{\varphi(c_2)}{az_2}\\
  \frac{\partial \varphi(c_1)}{\partial z_1} = \frac{\partial \varphi(c_1)}{\partial c_1}
  \frac{\partial c_1}{\partial z_1} = \frac{c_1 \varphi(c_1)}{az_1} &\qquad& 
  \frac{\partial \varphi(c_1)}{z_2} = \frac{\partial
    \varphi(c_1)}{\partial c_1} \frac{\partial c_1}{\partial z_2} = -
  \frac{c_1 \varphi(c_1)}{a z_2}\\
  \frac{\partial \varphi(c_2)}{\partial z_1} = \frac{\partial \varphi(c_2)}{\partial c_2}
  \frac{\partial c_2}{\partial z_1} = - \frac{c_2 \varphi(c_2)}{a z_1} &\qquad&
  \frac{\partial \varphi(c_2)}{\partial z_2} = \frac{\partial \varphi(c_2)}{\partial c_2}
  \frac{\partial c_2}{\partial z_2} = \frac{c_2 \varphi(c_2)}{a z_2}
\end{eqnarray}

Define
\begin{equation}
  \label{eq:3}
  A = \frac{1}{z_1}\Phi(c_1) \quad \text{and} \quad B = \frac{1}{z_2}\Phi(c_2)
\end{equation}
Consequently, $F(z_1, z_2) = exp(-A -B)$ and
\begin{equation}
  \label{eq:4}
  \frac{\partial F}{\partial z_1} (z_1, z_2) = - \left(\frac{\partial
      A}{\partial z_1} + \frac{\partial B}{\partial z_1} \right)
  F(z_1, z_2)\qquad
  \frac{\partial F}{\partial z_2} (z_1, z_2) = - \left(\frac{\partial
      A}{\partial z_2} + \frac{\partial B}{\partial z_2} \right)
  F(z_1, z_2)
\end{equation}
By noting that
\begin{eqnarray}
  \label{eq:5}
  \frac{\partial A}{\partial z_1} &=& -\frac{\Phi(c_1)}{z_1^2} +
  \frac{1}{z_1} \left(-\frac{\varphi(c_1)}{az_1}\right) =
  -\frac{\Phi(c_1)}{z_1^2} - \frac{\varphi(c_1)}{az_1^2}\\
  \frac{\partial B}{\partial z_1} &=& \frac{1}{z_2}
  \frac{\varphi(c_2)}{az_1} = \frac{\varphi(c_2)}{az_1z_2}\\
  \frac{\partial A}{\partial z_2} &=& \frac{1}{z_1}
  \frac{\varphi(c_1)}{az_2} = \frac{\varphi(c_1)}{az_1z_2}\\
  \frac{\partial B}{\partial z_2} &=& -\frac{\Phi(c_2)}{z_2^2} +
  \frac{1}{z_2} \left(- \frac{\varphi(c_2)}{az_2}\right) =
  -\frac{\Phi(c_2)}{z_2^2} - \frac{\varphi(c_2)}{az_2^2}
\end{eqnarray}
and
\begin{eqnarray}
  \label{eq:10}
   \frac{\partial^2 A}{\partial z_2 \partial z_1} &=& 
   \frac{\partial }{\partial z_2} \left(-\frac{\Phi(c_1)}{z_1^2} -
     \frac{\varphi(c_1)}{az_1^2}\right) = -\frac{\varphi(c_1)}{a z_1^2
     z_2} + \frac{c_1\varphi(c_1)}{a^2 z_1^2 z_2} = -\frac{c_2
     \varphi(c_1)}{a^2z_1^2z_2}\\
   \frac{\partial^2 B}{\partial z_2 \partial z_1} &=& \frac{\partial
   }{\partial z_2} \frac{\varphi(c_2)}{az_1z_2} =
   -\frac{c_1\varphi(c_2)}{a^2z_1z_2^2}
\end{eqnarray}
So that,
\begin{eqnarray}
  \label{eq:7}
  \frac{\partial F}{\partial z_1} (z_1, z_2) &=&  \left(
    \frac{\Phi(c_1)}{z_1^2} + \frac{\varphi(c_1)}{az_1^2} -
    \frac{\varphi(c_2)}{az_1z_2} \right) F(z_1, z_2)\\
  \frac{\partial F}{\partial z_2} (z_1, z_2) &=& \left(
   \frac{\Phi(c_2)}{z_2^2} + \frac{\varphi(c_2)}{az_2^2}
   -\frac{\varphi(c_1)}{az_1z_2} \right) F(z_1, z_2)
\end{eqnarray}
Finally,
\begin{equation}
  \label{eq:9}
  \frac{\partial^2 F}{\partial z_2 \partial z_1} (z_1,
  z_2) = - \left(\frac{\partial^2 A}{\partial z_2 \partial z_1} +
    \frac{\partial^2 B}{\partial z_2 \partial z_1} \right) F(z_1, z_2)
  - \left(\frac{\partial A}{\partial z_1} + \frac{\partial B}{\partial
      z_1} \right) \frac{\partial F}{\partial z_2} (z_1, z_2)
\end{equation}
Thus, it leads to the following relation:
\begin{equation}
  \label{eq:11}
  \frac{f(z_1, z_2)}{F(z_1, z_2)} = \frac{c_2 \varphi(c_1)}{a^2z_1^2z_2} + \frac{c_1
    \varphi(c_2)}{a^2z_1z_2^2} + \left(\frac{\Phi(c_1)}{z_1^2} +
    \frac{\varphi(c_1)}{az_1^2} - \frac{\varphi(c_2)}{az_1z_2} \right)
  \left(\frac{\Phi(c_2)}{z_2^2} + \frac{\varphi(c_2)}{az_2^2} -
    \frac{\varphi(c_1)}{az_1z_2} \right)
\end{equation}

\subsection{Gradient computation}
\label{sec:gradient-computation}

The gradient of the density must be known as we fit our model using
pairwise likelihood rather than the ``full'' likelihood. Consequently,
our model is ``mispecified'' and we need to compute standard errors
using a \textbf{sandwich estimator}. Let $\hat{\theta}$ be the maximum
pairwise likelihood estimate; then:
\begin{equation*}
  \hat{\theta} \sim \mathcal{N}\left(\theta, H^{-1} J H^{-1}\right)
\end{equation*}
where $H$ is the Fisher information matrix and $J$ the gradient of the
log pairwise likelihood.

Let us recall that the log pairwise likelihood is defined by:
\begin{equation*}
  \ell_{pair}(\mathbf{x}, \Sigma^{-1}) = \sum_{k = 1}^{n_{obs}}
  \sum_{i=1}^{n_{site}-1} \sum_{j=i+1}^{n_{site}} \log f(x_k^{(i)}, x_k^{(j)})
\end{equation*}
where $n_{obs}$ is the number of observations, $\mathbf{x}_k =
(x_k^{(1)}, \ldots, x_k^{(n_{site})})$ is the $k$-th observation vector,
$n_{site}$ is the number of site within the region and $f$ is the
bivariate density.

Consequently, the gradient of the log pairwise density is given by:
\begin{equation*}
  \nabla f_{pair}(\mathbf{x}, \Sigma^{-1}) = \sum_{i=1}^{n_{site}-1}
  \sum_{j=i+1}^{n_{site}} \nabla \log f(x_k^{(i)}, x_k^{(j)})
\end{equation*}

Let define:
\begin{eqnarray*}
  A &=& - \frac{\Phi(c1)}{z_1} - \frac{\Phi(c2)}{z_2}\\
  B &=& \frac{\Phi(c_2)}{z_2^2} + \frac{\varphi(c_2)}{az_2^2} -
  \frac{\varphi(c_1)}{az_1z_2}\\
  C &=& \frac{\Phi(c_1)}{z_1^2} + \frac{\varphi(c_1)}{az_1^2} -
  \frac{\varphi(c_2)}{az_1z_2}\\
  D &=& \frac{c_2 \varphi(c_1)}{a^2z_1^2z_2} + \frac{c_1
    \varphi(c_2)}{a^2z_1z_2^2}  
\end{eqnarray*}
so that,
\begin{equation*}
  \log f(x_k^{(i)}, x_k^{(j)}) = A + log(B C + D)
\end{equation*}

As the logarithm of the bivariate density $f$ is only a function of the
mahalanobis distance $a$, the gradient is given through the following
relation\footnote{algebra operators are defined componentwise.}:
\begin{equation*}
  \nabla \log f(x_k^{(i)}, x_k^{(j)}) = \frac{\partial}{\partial a}
  \log f(x_k^{(i)}, x_k^{(j)}) \left( \frac{\partial a}{\partial
      cov{11}}, \frac{\partial a}{\partial cov{12}}, \frac{\partial
      a}{\partial cov{22}} \right)^T
\end{equation*}

For clarity purposes, we first compute the following quantities:
\begin{eqnarray*}
  \frac{\partial c_1}{\partial a} = \frac{1}{2} - \frac{1}{a^2} \log
  \frac{z_2}{z_1} = \frac{c_2}{a} &\qquad& \frac{\partial c_2}{\partial
    a} = \frac{c_1}{a}\\
  \frac{\partial \Phi(c_1)}{\partial a} = \frac{\partial
    \Phi(c_1)}{\partial c_1} \frac{\partial c_1}{\partial a} =
  \frac{c_2 \varphi(c_1)}{a} &\qquad& \frac{\partial
    \Phi(c_2)}{\partial a} = \frac{c_1 \varphi(c_2)}{a}\\
  \frac{\partial \varphi(c_1)}{\partial a} = \frac{\partial
    \varphi(c_1)}{\partial c_1} \frac{\partial c_1}{\partial a} =
  -\frac{c_1c_2 \varphi(c_1)}{a} &\qquad& \frac{\partial
    \varphi(c_2)}{\partial a} = -\frac{c_1c_2 \varphi(c_2)}{a}\\
  \frac{\partial c_2\varphi(c_1)}{\partial a} = \frac{c_1(1 -
    c_2^2)\varphi(c_1)}{a} &\qquad& \frac{\partial
    c_1\varphi(c_2)}{\partial a} = \frac{(1-c_1^2)c_2\varphi(c_2)}{a}
\end{eqnarray*}

Consequently, we have:
\begin{eqnarray*}
  dA &=& \frac{\partial A}{\partial a} = - \frac{1}{z_1} \frac{c_2
    \varphi(c_1)}{a} - \frac{1}{z_2} \frac{c_1 \varphi(c_2)}{a} =
  -\frac{c_2 \varphi(c_1)}{az_1} - \frac{c_1 \varphi(c_2)}{az_2}\\
  dC &=& \frac{\partial C}{\partial a} = \frac{1}{z_1^2} \frac{c_2
    \varphi(c_1)}{a} + \frac{1}{z_1^2}
  \frac{-\frac{c_1c_2\varphi(c_1)}{a} a  - \varphi(c_1)}{a^2} -
  \frac{1}{z_1z_2} \frac{-\frac{c_1c_2 \varphi(c_2)}{a}a -
    \varphi(c_2)}{a^2}\\
  &=& \frac{c_2 \varphi(c_1)}{az_1^2} -
  \frac{(1+c_1c_2)\varphi(c_1)}{a^2z_1^2} +
  \frac{(1+c_1c_2)\varphi(c_2)}{a^2z_1z_2}\\
  &=& \frac{\left[c_2(a - c_1)-1\right] \varphi(c_1)}{a^2z_1^2} +
  \frac{(1+c_1c_2)\varphi(c_2)}{a^2z_1z_2}\\
  &=& \frac{(c_2^2 - 1) \varphi(c_1)}{a^2z_1^2} +
  \frac{(1+c_1c_2)\varphi(c_2)}{a^2z_1z_2}\\
  dB &=& \frac{\partial B}{\partial a} = \frac{(c_1^2 - 1)
    \varphi(c_2)}{a^2z_2^2} +
  \frac{(1+c_1c_2)\varphi(c_1)}{a^2z_1z_2}\\
  dD &=& \frac{\partial D}{\partial a} = \frac{1}{z_1^2z_2}\frac{\frac{c_1(1 -
      c_2^2)\varphi(c_1)}{a}a^2 - 2a c_2\varphi(c_1)}{a^4} +
  \frac{1}{z_1z_2^2}\frac{\frac{(1-c1^2)c_2\varphi(c_2)}{a}a^2 - 2a
    c_1\varphi(c_2)}{a^4}\\
 &=& \frac{(c_1- 2 c_2 - c_1c_2^2) \varphi(c_1)}{a^3z_1^2z_2} +
 \frac{(c_2- 2 c_1 - c_1^2c_2) \varphi(c_2)}{a^3z_1z_2^2}
\end{eqnarray*}

Finally,
\begin{equation*}
  \nabla \log f(x_k^{(i)}, x_k^{(j)}) = \left[dA + \frac{(C dB + B dC
    +dD)}{BC + D} \right] \left( \frac{\partial a}{\partial cov{11}},
    \frac{\partial a}{\partial cov{12}}, \frac{\partial a}{\partial
      cov{22}} \right)^T
\end{equation*}

\section{The Schlather Characterisation}
\label{sec:schlather-char}

The Schlather characterisation of a max-stable process is given by:

\begin{equation}
  \label{eq:smith}
  \Pr[Z_1 \leq z_1, Z_2 \leq z_2] = \exp\left[-\frac{1}{2}
    \left(\frac{1}{z_1} + \frac{1}{z_1} \right) \left(1 + \sqrt{1 - 2
        (\rho(h) + 1) \frac{z_1 z_2}{(z_1 + z_2)^2}} \right) \right]
\end{equation}
where $h$ is the distance between location \#1 and location \#2 and
$\rho(h)$ is a valid correlation function.

\subsection{Useful quantities}
\label{sec:usefull-quantities}

Computation of the density as well as the gradient of the density is
not difficult but ``heavy'' though. For computation facilities and to
help readers, we define:
\begin{equation}
  \label{eq:1}
  c_1 = 2 \left(\rho(h) + 1 \right) \frac{z_1z_2}{(z_1+z_2)^2} \quad
  \text{and} \quad c_2 = \sqrt{1 - c1} \quad \text{and} \quad
  c_3 = \frac{1}{z_1} + \frac{1}{z_2}
\end{equation}

With these notations, we have:
\begin{equation*}
  \Pr[Z_1 \leq z_1, Z_2 \leq z_2] = \exp\left[-\frac{c_3 (1 +
    c_2)}{2} \right]
\end{equation*}

\subsection{Density computation}
\label{sec:density-computation}

For ease of computation, first we need to compute the following
partial derivatives:
\begin{eqnarray*}
  \frac{\partial c_2}{\partial z_1} &=& \frac{2
    \left(\rho(h)+1\right)z_2}{(z_1+z_2)^2} - \frac{4
    \left(\rho(h)+1\right)z_1z_2}{(z_1+z_2)^3} =  \frac{c1}{z_1} -
  \frac{2 c_1}{z_1+z_2} = \frac{c_1 (z_2 - z_1)}{z_1 (z_1 + z_2)}\\
  \frac{\partial c_2}{\partial z_2} &=& -\frac{c_1 (z_2 - z_1)}{z_2
    (z_1 + z_2)}\\
  \frac{\partial c_3}{\partial z_1} &=& -\frac{1}{z_1^2}\\
  \frac{\partial c_3}{\partial z_2} &=& -\frac{1}{z_2^2}\\
\end{eqnarray*}

Then
\end{document}

%%% Local Variables: 
%%% mode: latex
%%% TeX-master: t
%%% End: 
